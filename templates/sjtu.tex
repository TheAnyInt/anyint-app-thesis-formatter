\documentclass[UTF8,a4paper,12pt]{ctexart}
\usepackage{amsmath}
\usepackage{array}
\usepackage{caption}
\usepackage{amssymb}
\usepackage{color}
\usepackage{graphicx}
\usepackage{times}
\usepackage{mathptmx}
\usepackage{fancyhdr}
\usepackage{booktabs}
\usepackage[T1]{fontenc}
\usepackage{enumerate}
\usepackage{float}
\usepackage{setspace}
\usepackage{hyperref}
\hypersetup{hidelinks}
\usepackage{tocloft}
\usepackage{titletoc}
\usepackage{titlesec}

% 页面版心大小
\setlength{\textheight}{22cm}
\setlength{\textwidth}{15cm}
\setlength{\voffset}{-1.14cm}
\setlength{\hoffset}{-0.57cm}
\setlength{\headheight}{1cm}
\setlength{\topmargin}{0cm}
\setlength{\headsep}{1.8cm}
\setlength{\footskip}{1.2cm}

% 页眉页脚设置
\pagestyle{fancy}
\fancyhf{}
\fancyfoot[C]{\thepage}
\fancypagestyle{plain}{\pagestyle{fancy}}
\fancyhfinit{\small}
\renewcommand{\headrulewidth}{0.5pt}

% 行距
\setlength{\baselineskip}{20pt}

% 目录设置
\renewcommand{\cftsecleader}{\cftdotfill{\cftdotsep}}
\titlecontents{section}[0pt]
              {\addvspace{6pt}\filright\large\bf}
              {\contentspush{\thecontentslabel \quad }}
              {}{\titlerule*[8pt]{.}\contentspage}
\setlength{\cftbeforesubsecskip}{6pt}
\setlength{\cftsubsecindent}{1em}
\renewcommand{\cftsubsecfont}{\normalsize}

% 图表设置
\numberwithin{figure}{section}
\numberwithin{table}{section}
\renewcommand{\thefigure}{\thesection{}-\arabic{figure}}
\renewcommand{\thetable}{\thesection{}-\arabic{table}}
\captionsetup{font={small,bf},labelsep=quad,justification=centering}

% 各级标题格式
\ctexset{section={
  format={\heiti \zihao{3} \bfseries \center},
  number={第\chinese{section}章}
}}
\titlespacing*{\section}{0pt}{24pt}{18pt}
\titlespacing{\subsection}{0pt}{24pt}{12pt}
\titlespacing{\subsubsection}{0pt}{12pt}{6pt}
\titleformat*{\subsection}{\heiti\large\bfseries}
\titleformat*{\subsubsection}{\heiti\normalsize\bfseries}

\begin{document}

% ===== 封面 =====
\thispagestyle{empty}
\begin{center}
\vspace*{1.5cm}
{\zihao{-0}\heiti 上海交通大学}
\vspace{0.8cm}

{\zihao{2}\heiti 本科毕业论文}
\vspace{2.5cm}

{\zihao{2}\heiti {{{metadata.title}}} }
\vspace{0.5cm}

{{#metadata.title_en}}
{\zihao{3} {{{metadata.title_en}}} }
{{/metadata.title_en}}
\vspace{3cm}

{\zihao{4}
\renewcommand{\arraystretch}{1.5}
\begin{tabular}{rl}
学生姓名: & {{{metadata.author_name}}} \\
{{#metadata.student_id}}学生学号: & {{{metadata.student_id}}} \\{{/metadata.student_id}}
{{#metadata.major}}专\hspace{2em}业: & {{{metadata.major}}} \\{{/metadata.major}}
{{#metadata.supervisor}}指导教师: & {{{metadata.supervisor}}} \\{{/metadata.supervisor}}
{{#metadata.school}}学\hspace{2em}院: & {{{metadata.school}}} \\{{/metadata.school}}
\end{tabular}
}
\vspace{2.5cm}

{{#metadata.date}}
{\zihao{4} {{{metadata.date}}} }
{{/metadata.date}}
\end{center}
\newpage

% ===== 摘要 =====
\pagenumbering{Roman}
\fancyhead[L]{上海交通大学学位论文}
\fancyhead[R]{}

{{#abstract}}
\phantomsection
\addcontentsline{toc}{section}{摘要}
\begin{center}
{\zihao{3}\heiti 摘\quad 要}
\end{center}
\vspace{1cm}

{{{abstract}}}

\vspace{1cm}
{{#keywords}}
\noindent\textbf{关键词:}{{{keywords}}}
{{/keywords}}
\newpage
{{/abstract}}

% ===== 英文摘要 =====
{{#abstract_en}}
\phantomsection
\addcontentsline{toc}{section}{ABSTRACT}
\begin{center}
{\zihao{3}\bfseries ABSTRACT}
\end{center}
\vspace{1cm}

{{{abstract_en}}}

\vspace{1cm}
{{#keywords_en}}
\noindent\textbf{Keywords:} {{{keywords_en}}}
{{/keywords_en}}
\newpage
{{/abstract_en}}

% ===== 目录 =====
\phantomsection
\addcontentsline{toc}{section}{目录}
\tableofcontents
\newpage

% ===== 正文(动态章节)=====
\pagenumbering{arabic}

{{#sections}}
{{#isLevel1}}\section{ {{{title}}} }{{/isLevel1}}
{{#isLevel2}}\subsection{ {{{title}}} }{{/isLevel2}}
{{#isLevel3}}\subsubsection{ {{{title}}} }{{/isLevel3}}
{{{content}}}

{{/sections}}

% ===== 图表(如有)=====
{{#figures}}
\begin{figure}[H]
\centering
\includegraphics[width=0.8\textwidth]{ {{{filename}}} }
\caption{图{{{index}}}}
\label{ {{{label}}} }
\end{figure}
{{/figures}}

% ===== 参考文献 =====
{{#references}}
\newpage
\phantomsection
\addcontentsline{toc}{section}{参考文献}
\begin{center}
{\zihao{3}\heiti 参考文献}
\end{center}
\vspace{1cm}

{{{references}}}
{{/references}}

% ===== 致谢 =====
{{#acknowledgements}}
\newpage
\phantomsection
\addcontentsline{toc}{section}{致谢}
\begin{center}
{\zihao{3}\heiti 致\quad 谢}
\end{center}
\vspace{1cm}

{{{acknowledgements}}}
{{/acknowledgements}}

\end{document}
